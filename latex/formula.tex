%%%%%%%%%%%%%%%%%%%%%%%%%%%%%%%%%%%%%%%%%%%%%%%%%%%%%%%%%%%%%%%%%%%%%%%%%%%%%%%%
\documentclass[
%a4paper,       % letterpaper, a4paper, a5paper
%10pt,              % 10pt, 11pt, 12pt
reprint,           % journal's actual layout
%preprint,          % 12pt, single-column
superscriptaddress,% authors with affiliations via superscripts
amsmath,           % add AMS-Latex features
amssymb,           % add extra AMS symbols, including amsfonts
aps,               % aps or aip
prl,               % prl, pra, prb, prc, prd, pre, prstab
%showpacs,          % make PACS codes appear
notitlepage,       % control appearance of title page
%longbibliography,  % show  article titles in the bibliography
floatfix,          % process floats as early as possible
%showkeys,          % option to make keywords appear
%titlepage,         %
%eqsecnum,          % number equations by section
nofootinbib,
onecolumn
]{revtex4-1}

%%%%%%%%%%%%%%%%%%%%%%%%%%%%%%% load packages %%%%%%%%%%%%%%%%%%%%%%%%%%%%%%%%%%
\usepackage{tensor}     % manipulate tensors
\usepackage{graphicx}   % include figures
\usepackage[
colorlinks=true,        % color link
citecolor=blue,         % cite color
linkcolor=blue,         % link color
urlcolor=blue           % url color
]{hyperref}             % create hyperlinks
\usepackage{bm}         % \bm{<text>} Bold math symbols
\usepackage{xcolor}     % textcolor
%\usepackage{dcolumn}    % Align table columns on decimal point
%\usepackage{array}      % dcolumn depends on array
\usepackage{tabulary}
\usepackage{color}      % \textcolor{declared-color}{text}
%\usepackage{longtable}  % large table\
\usepackage[utf8]{inputenc} % accented characters for .bib file using XeLaTex
\usepackage[section]{placeins} % figures processing

%%%%%%%%%%%%%%%%%%%%%%%%%%%%%%%%% new commands %%%%%%%%%%%%%%%%%%%%%%%%%%%%%%%%%
\newcommand{\nc}{\newcommand*} 

%%%%%%%%%%%%%%%%%%%%%%%%%%%%%%%%%% alphabet %%%%%%%%%%%%%%%%%%%%%%%%%%%%%%%%%%%%
\nc{\al}{\alpha}
\nc{\s}{\sigma}
\nc{\kp}{\kappa}
\nc{\dt}{\delta}
\nc{\Dt}{\Delta}
\nc{\Ld}{\Lambda}
\nc{\p}{\partial}
\nc{\Gm}{\Gamma}
\nc{\om}{\omega}
\nc{\Om}{\Omega}
\nc{\rd}{\mathrm{d}}
\nc{\Od}[1]{\mathcal{O}(#1)} % order operator
\nc{\one}{\uppercase\expandafter{\romannumeral1}}
\nc{\two}{\uppercase\expandafter{\romannumeral2}}
\nc{\three}{\uppercase\expandafter{\romannumeral3}}
%%%%%%%%%%%%%%%%%%%%%%%%%%%%%%%%%%% equations %%%%%%%%%%%%%%%%%%%%%%%%%%%%%%%%%%
\def\({\left(}
\def\){\right)}
\def\[{\left[}
\def\]{\right]}
\def\e{\begin{equation}}
\def\q{\end{equation}}
\def\m{\begin{eqnarray}}
\def\n{\end{eqnarray}}
%%%%%%%%%%%%%%%%%%%%%%%%%%%%%%%%%%%%%%%%%%%%%%%%%%%%%%%%%%%%%%%%%%%%%%%%%%%%%%%%\
%%%%%%%%%%%%%%%%%%%%%%%%%%%%%%%%%% citations %%%%%%%%%%%%%%%%%%%%%%%%%%%%%%%%%%%
\nc{\Eq}[1]{Eq.~\eqref{#1}}     % equation
\nc{\Fig}[1]{Fig.~\ref{#1}}     % figure
\nc{\Table}[1]{Table~\ref{#1}}  % table
\nc{\Sec}[1]{Sec.~\ref{#1}}     % section
%%%%%%%%%%%%%%%%%%%%%%%%%%%%%%%%%%%%%%%%%%%%%%%%%%%%%%%%%%%%%%%%%%%%%%%%%%%%%%%%
%%%%%%%%%%%%%%%%%%%%%%%%%%%%%%% gravitational waves %%%%%%%%%%%%%%%%%%%%%%%%%%%%
\nc{\Msun}{M_\odot}             % solar mass
\nc{\fpbh}{f_{\mathrm{pbh}}}    % f_pbh
\nc{\fpbhn}{f_{\mathrm{pbh0}}}    % f_pbh
\nc{\mR}{\mathcal{R}} % merger rate density
\nc{\seq}{\sigma_{\mathrm{eq}}}
\nc{\ogw}{\Omega_{\mathrm{GW}}}
\nc{\gpcyr}{\mathrm{Gpc}^{-3}\,\mathrm{yr}^{-1}}
\nc{\lvc}{LIGO/Virgo} % LIGO-VIRGO collaboration
\nc{\SNR}{\mathrm{SNR}} % signal to noise ratio
\nc{\mmin}{{m_{\mathrm{min}}}}
\nc{\mmax}{{m_{\mathrm{max}}}}
\nc{\Mmin}{{M_{\mathrm{min}}}}
\nc{\fmin}{{f_{\mathrm{min}}}}
\nc{\VT}{\mathrm{VT}}
\nc{\rhoGW}{\rho_{\mathrm{GW}}}
\nc{\vth}{\vec{\theta}}
\nc{\vd}{\vec{d}}
\nc{\vla}{\vec{\lambda}}
\nc{\av}[1]{\langle #1 \rangle} % average bracket
\nc{\fyr}{f_{yr}}
%%%%%%%%%%%%%%%%%%%%%%%%%%%%%%%%%%%%%%%%%%%%%%%%%%%%%%%%%%%%%%%%%%%%%%%%%%%%%%%%
%%%%%%%%%%%%%%%%%%%%%%%%%%%%%%%%%%%%% other %%%%%%%%%%%%%%%%%%%%%%%%%%%%%%%%%%%%
\nc{\addref}{[\textcolor{red}{add ref}] } % placeholder of references
\nc{\eg}{\textit{e.g.~}}
\nc{\app}{\approx}
\nc{\hf}{\frac{1}{2}}
\nc{\discuss}{\textcolor{red}{Add discussion here!}}
\nc{\red}[1]{\textcolor{red}{#1}}
%%%%%%%%%%%%%%%%%%%%%%%%%%%%%%%%%%%%%%%%%%%%%%%%%%%%%%%%%%%%%%%%%%%%%%%%%%%%%%%%
%%%%%%%%%%%%%%%%%%%%%%%%%%% only used in this paper %%%%%%%%%%%%%%%%%%%%%%%%%%%%
\nc{\hp}{h_+} % h plus
\nc{\hc}{h_{\times}} % h cross
\nc{\Oh}{\hat{\Omega}}
\nc{\vx}{\vec{x}}
\nc{\mh}{\hat{m}}
\nc{\nh}{\hat{n}}
\nc{\zh}{\hat{z}}
\nc{\ph}{\hat{p}}
%%%%%%%%%%%%%%%%%%%%%%%%%%%%%%%%%%%%%%%%%%%%%%%%%%%%%%%%%%%%%%%%%%%%%%%%%%%%%%%%

%%%%%%%%%%%%%%%%%%%%%%%%%%%%%%%%%%%%%%%%%%%%%%%%%%%%%%%%%%%%%%%%%%%%%%%%%%%%%%%%
\begin{document}
%%%%%%%%%%%%%%%%%%%%%%%%%%%%%%%%%%%%%%%%%%%%%%%%%%%%%%%%%%%%%%%%%%%%%%%%%%%%%%%%
	
%%%%%%%%%%%%%%%%%%%%%%%%%%%%%%%%%%%% title %%%%%%%%%%%%%%%%%%%%%%%%%%%%%%%%%%%%%
\title{Some famous formula for pulsar timing array (PTA)}
	
%%%%%%%%%%%%%%%%%%%%%%%%%%%%%%%%%%%% author %%%%%%%%%%%%%%%%%%%%%%%%%%%%%%%%%%%%
\author{Zu-Cheng Chen}
\email{chenzucheng@itp.ac.cn} %bingining@gmail.com
\affiliation{CAS Key Laboratory of Theoretical Physics, 
Institute of Theoretical Physics, Chinese Academy of Sciences,
Beijing 100190, China}
\affiliation{School of Physical Sciences, 
University of Chinese Academy of Sciences, 
No. 19A Yuquan Road, Beijing 100049, China}
	
%%%%%%%%%%%%%%%%%%%%%%%%%%%%%%%%%%%%% date %%%%%%%%%%%%%%%%%%%%%%%%%%%%%%%%%%%%%
\date{\today}

%%%%%%%%%%%%%%%%%%%%%%%%%%%%%%%%% abstract %%%%%%%%%%%%%%%%%%%%%%%%%%%%%%%%%%%%%
\begin{abstract}
    

\end{abstract}
	
	
\maketitle
	
%%%%%%%%%%%%%%%%%%%%%%%%%%%%%%%%%%%%%%%%%%%%%%%%%%%%%%%%%%%%%%%%%%%%%%%%%%%%%%%%
\section{Background}
We consider a general metric
\e 
    g_{ab} = \eta_{ab} + h_{ab},
\q 
where 
\e 
    \eta_{ab} = \begin{pmatrix}
    -1 & 0 & 0 & 0\\
    0 & 1 & 0 &0\\
    0 & 0 & 1 &0\\
    0 & 0 & 0 & 1
\end{pmatrix}
\q
is the Minkowski metric, and $h_{ab}$
is the perturbation of the spacetime or gravitational wave (GW).
A metric perturbation in a spatial transverse and traceless gauge has a plane
wave expansion given by 
\e 
    h_{ab}(t, \vec{x}) = \sum_{A} \int_{-\infty}^{\infty} df \int d\Oh\, 
        e^{i2\pi f(t-\Oh\cdot\vec{x})} h_{A}(f, \Oh) e_{ij}^A(\Oh),
\q 
where $f$ is the frequency of the GWs, $\vec{k}=2\pi f \Oh$ is the wave vector,
$\Oh$ is a unit vector that points along the direction of travel of the waves,
$i,j=x, y, z$ are spatial indices, and the index $A=(+,\times,b,l,x,y)$ labels polarization
of GWs.
The polarization tensors $e^A_{ij}(\Oh)$ are defined as
\m 
  e^{+}_{ij} &=& \mh_i \mh_j - \nh_i \nh_j, \\
  e^{\times}_{ij} &=& \mh_i \nh_j + \nh_i \mh_j,\\
  e^{b}_{ij} &=& \mh_i \mh_j + \nh_i \nh_j, \\
  e^{l}_{ij} &=& \Oh_i \Oh_j, \\
  e^{x}_{ij} &=& \mh_i \Oh_j + \Oh_i \mh_j, \\
  e^{y}_{ij} &=& \nh_i \Oh_j + \Oh_i \nh_j,
\n
where
\m 
    \Oh &=& (\sin\theta \cos\phi, \sin\theta \sin\phi, \cos\theta)\\
    \mh &=& (\sin\phi, -\cos\phi, 0),\\
    \nh &=& (\cos\theta \cos\phi, \cos\theta \sin\phi, -\sin\theta).
\n 
It is easy to verify that $\Oh$, $\mh$ and $\nh$ are perpendicular to each other.
Furthermore, the polarization tensors are normalized as
\e 
    e^A_{ij} e^{Bij} = 2 \dt^{AB}.
\q 

Now consider the metric perturbation from a single gravitational wave traveling
along the $z$-axis so that $\Oh=\zh=(0,0,1)$, $\theta=0$ and $\phi=\pi/2$.
In this case $\mh = \hat{x} = (1,0,0)$ and $\nh = \hat{y} = (0,1,0)$.
Besides,
\m 
    e^{+}_{ij}(\zh) &=& \mh_i \mh_j - \nh_i \nh_j\\
        &=& \begin{pmatrix}
            1\\
            0\\
            0
        \end{pmatrix} \(1, 0, 0\) - \begin{pmatrix}
            0\\
            1\\ 
            0
        \end{pmatrix} \(0, 1, 0\)\\
        &=& \begin{pmatrix}
            1 & 0 & 0\\
            0 & 0 & 0\\
            0 & 0 & 0
        \end{pmatrix} - \begin{pmatrix}
        0 & 0 & 0\\
        0 & 1 & 0\\
        0 & 0 & 0
        \end{pmatrix}
        = \begin{pmatrix}
            1 & 0 & 0\\
            0 & -1 & 0\\
            0 & 0 & 0
        \end{pmatrix},
\n
and
\m 
e^{\times}_{ij}(\zh) &=& \mh_i \nh_j + \nh_i \mh_j\\
&=& \begin{pmatrix}
    1\\
    0\\
    0
\end{pmatrix} \(0, 1, 0\) + \begin{pmatrix}
    0\\
    1\\ 
    0
\end{pmatrix} \(1, 0, 0\)\\
&=& \begin{pmatrix}
    0 & 1 & 0\\
    0 & 0 & 0\\
    0 & 0 & 0
\end{pmatrix} + \begin{pmatrix}
    0 & 0 & 0\\
    1 & 0 & 0\\
    0 & 0 & 0
\end{pmatrix}
= \begin{pmatrix}
    0 & 1 & 0\\
    1 & 0 & 0\\
    0 & 0 & 0
\end{pmatrix}.
\n

The metric perturbation is given explicitly by
\m 
    h_{ij}(t-z) &\equiv& h_{ij}(t, \Oh=\zh)\\
        &=& \sum_A \int_{-\infty}^{\infty} df
        e^{i2\pi f(t-z)} h_A(f, \zh) e^A_{ij}(\zh)\\
        &=& \sum_A h_A(f, t-z) e^A_{ij}(\zh)\\
        &=& \sum_A h_A e^A_{ij}(\zh)
        = \hp e^+_{ij}(\zh) + \hc e^\times_{ij}(\zh)\\
        &=& \begin{pmatrix}
            \hp & 0 & 0\\
            0 & -\hp & 0\\
            0 & 0 & 0
        \end{pmatrix} + \begin{pmatrix}
        0 & \hc & 0\\
        \hc & 0 & 0\\
        0 & 0 & 0
    \end{pmatrix}
= \begin{pmatrix}
    \hp & \hc & 0\\
    \hc & -\hp & 0\\
    0 & 0 & 0
\end{pmatrix}.
\n 
since
\e 
    \Oh\cdot\vx = \zh\cdot\vx = z,
\q 
and we have defined 
\e
    h_A = h_A(f, t-z) 
        \equiv \int_{-\infty}^{\infty} df e^{i2\pi f(t-z)} h_A(f, \zh).
\q
Therefore the physical metric due to the perturbation is given by
\e 
g_{ab} = \eta_{ab} + h_{ab}(t-z) = \begin{pmatrix}
    -1 & 0 & 0 & 0\\
    0 & 1+\hp & \hc &0\\
    0 & \hc & 1-\hp &0\\
    0 & 0 & 0 & 1
\end{pmatrix}.
\q

\section{Correlation}
The antenna patterns are defined as
\e 
    F^A(\Oh) = e^A_{ij}(\Oh) \frac{\ph^i\ph^j}{2(1+\Oh\cdot\ph)}.
\q 
The overlap function for two pulsars are 
\e 
    \Gm_{ab}(|f|) = \sum_A \Gm^A_{ab}(|f|),
\q 
where 
\e 
    \Gm^A_{ab}(|f|) = \frac{3}{4\pi} \int d\Oh \(e^{2\pi i f L_a(1+\Oh\cdot\ph_a)}-1\)
        \(e^{2\pi i f L_b(1+\Oh\cdot\ph_b)}-1\) F^A_a(\Oh) F^A_b(\Oh).
\q 
We now define
\m 
    \Gm^{TT}_{ab}(|f|) &=& \Gm^{+}_{ab}(|f|) + \Gm^{\times}_{ab}(|f|),\\
    \Gm^{ST}_{ab}(|f|) &=& \Gm^{b}_{ab}(|f|),\\
    \Gm^{VL}_{ab}(|f|) &=& \Gm^{x}_{ab}(|f|) + \Gm^{y}_{ab}(|f|),\\
    \Gm^{SL}_{ab}(|f|) &=& \Gm^{l}_{ab}(|f|).
\n 
Then the cross-spectral density is 
\m 
    S_{ab}(f) &=& \frac{1}{24\pi^2f^3} \frac{1+\kappa^2}{1+\kappa^2\(\frac{f}{\fyr}\)^{-\frac{2}{3}}}
        \[\Gm^{TT}_{ab} A_{TT}^2 \(\frac{f}{\fyr}\)^{-\frac{4}{3}}
        +\(\Gm^{ST}_{ab} A_{ST}^2 + \Gm^{VL}_{ab} A_{VL}^2 
            + \Gm^{SL}_{ab} A_{SL}^2\)\(\frac{f}{\fyr}\)^{-2}\]\\
        &=& \Gm^{TT}_{ab} S_{TT} + \Gm^{ST}_{ab}S_{ST} + \Gm^{VL}_{ab} S_{VL} 
        + \Gm^{SL}_{ab} S_{SL},
\n 
where
\m 
    S_{TT} = \frac{1}{24\pi^2f^3} \frac{1+\kappa^2}{1+\kappa^2\(\frac{f}{\fyr}\)^{-\frac{2}{3}}} A_{TT}^2 \(\frac{f}{\fyr}\)^{-\frac{4}{3}}, \\
    S_{ST} = \frac{1}{24\pi^2f^3} \frac{1+\kappa^2}{1+\kappa^2\(\frac{f}{\fyr}\)^{-\frac{2}{3}}} A_{ST}^2 \(\frac{f}{\fyr}\)^{-2}, \\
    S_{VL} = \frac{1}{24\pi^2f^3} \frac{1+\kappa^2}{1+\kappa^2\(\frac{f}{\fyr}\)^{-\frac{2}{3}}} A_{VL}^2 \(\frac{f}{\fyr}\)^{-2}, \\
    S_{SL} = \frac{1}{24\pi^2f^3} \frac{1+\kappa^2}{1+\kappa^2\(\frac{f}{\fyr}\)^{-\frac{2}{3}}} A_{SL}^2 \(\frac{f}{\fyr}\)^{-2}.
\n 
%%%%%%%%%%%%%%%%%%%%%%%%%%%%%%%%%%%%%%%%%%%%%%%%%%%%%%%%%%%%%%%%%%%%%%%%%%%%%%%%
%%%%%%%%%%%%%%%%%%%%%%%%%%%%%%%%%% references %%%%%%%%%%%%%%%%%%%%%%%%%%%%%%%%%%
%%%%%%%%%%%%%%%%%%%%%%%%%%%%%%%%%%%%%%%%%%%%%%%%%%%%%%%%%%%%%%%%%%%%%%%%%%%%%%%%
%\bibliographystyle{apj}
%\bibliography{./TTgaugeref}
	
%%%%%%%%%%%%%%%%%%%%%%%%%%%%%%%%%%%%%%%%%%%%%%%%%%%%%%%%%%%%%%%%%%%%%%%%%%%%%%%%
\end{document}
%%%%%%%%%%%%%%%%%%%%%%%%%%%%%%%%%%%%%%%%%%%%%%%%%%%%%%%%%%%%%%%%%%%%%%%%%%%%%%%%